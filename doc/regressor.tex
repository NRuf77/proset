\chapter{Prototype set regressor}
\label{ch_regressors}
%
The classifier model (\ref{eq_pkx}) is constructed using marginal probabilities as baseline and indicator functions as kernels on the target space.
If we replace these by probability density functions, we obtain a locally weighted regressor.
For our implementation, we rely entirely on Gaussians.
%
\section{Definition}
\label{sec_regressor_definition}
%
To derive the regression model, we follow the notation introduced in Section \ref{sec_classifier_definition} as far as possible.
The first major difference is that the random variables $Y_n$ are now presumed to be real-valued.
Our baseline distribution is defined in terms of the first two sample moments:
%
\begin{align}
\bar{y}_N&:=\frac{1}{N}\sum_{n=1}^Ny_n&
\hat{s}^2_N&:=\frac{1}{N-1}\sum_{n=1}^N(y_n-\bar{y}_N)^2
\label{eq_y_moments}
\end{align}
%
As kernel on the target space, we use the univariate Gaussian density parameterized in terms of its inverse standard deviation:
%
\begin{equation}
H_u(z):\R\rightarrow(0,\infty),z\mapsto H_u(z):=\frac{u}{\sqrt{2\pi}}\exp\left(-\frac{(uz)^2}{2}\right)\label{eq_kernel_2}
\end{equation}
%
In analogy to (\ref{eq_pkx}), we estimate the density of $Y$ conditional on $X$ as
%
\begin{equation}
f(Y=y|X=x)\approx\hat{f}(y|x):=
\frac{H_{\hat{s}^{-1}_N}(y-\bar{y}_N)+\sum_{b=1}^B\sum_{j=1}^{J_b}H_{u_b}(y-y_{s_{b,j}})w_{b,j}G_{v_b}(x-x_{s_{b,j}})}
{1+\sum_{b=1}^B\sum_{j=1}^{J_b}w_{b,j}G_{v_b}(x-x_{s_{b,j}})}
\label{eq_fyx}
\end{equation}
%
Here, we introduce an additional set of parameters $u_b>0$, $b\in\{1,\dots,B\}$, that control the smoothness of the density for each batch.
As we do not want these terms to vanish, the fitting procedure applies an $L_2$-penalty to the $u_b$, not elastic net regularization.
%
\begin{remark}
\begin{enumerate}
\item Unlike most regression algorithms that estimate a conditional mean, the proset regressor estimates a complete distribution.
It can capture properties like heteroskedasticity, multimodality, or skewness.
For any particular value of $x$, the model is equivalent to a Gaussian mixture.
%
\item A batch with $v_b\equiv0$ (independent of $x$) can still provide meaningful information on $Y$, effectively updating the baseline to be non-Gaussian.
\end{enumerate}
\end{remark}
%
\endinput
